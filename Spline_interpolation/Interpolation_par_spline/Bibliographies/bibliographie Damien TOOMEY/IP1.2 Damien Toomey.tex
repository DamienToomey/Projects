\documentclass{article}
\usepackage[T1]{fontenc}
\usepackage{graphicx}
\usepackage {subfigure}
\usepackage{color}
\usepackage{hyperref}
\usepackage{amsmath,amsfonts,amssymb}
\setlength{\parskip}{1em}
\usepackage{scrextend}
\usepackage{url}
\usepackage[margin=1in,footskip=0.25in]{geometry}

\title{Sources Bibliographiques}
\author{Damien TOOMEY}
\date{Decembre 2016}

\begin{document}

\maketitle

\vspace{10\baselineskip}
\begin{center}
\makeatother
\title{L'INTERPOLATION PAR SPLINE}\\
\end{center}

\vspace{12\baselineskip}
\begin{center}
A l'attention de M.Gleyse\\
\date{2016-2017}
\end{center}

\newpage
\begingroup\raggedleft
Les vid\'{e}os suivantes sont en anglais et ont \'{e}t\'{e} trouv\'{e}es sur Youtube.
\endgroup
\\\\
Interpolation lin\'{e}aire par spline th\'{e}orie
\\
{\color{blue}
\url{https://www.youtube.com/watch?v=KLUr1A6vyzs}}
\\
Interpolation lin\'{e}aire exemples
\\
{\color{blue}
\url{https://www.youtube.com/watch?annotation_id=annotation_996388&feature=iv&src_vid=KLUr1A6vyzs&v=0YCJS19JTv0}}
\\ 
L'auteur de ces deux vid\'{e}os explique ce qu'est l'interpolation lin\'{e}aire et montre qu'elle est limit\'{e}e pour deux raisons.
\\\\
Interpolation quadratique par spline th\'{e}orie Partie 1
\\
{\color{blue}
\url{https://www.youtube.com/watch?v=j_jBK7zJ1vU}}
\\ 
On relit deux points cons\'{e}cutifs par une portion de parabole.
\\\\
Interpolation quadratique par spline th\'{e}orie Partie 2
\\
{\color{blue}
\url{https://www.youtube.com/watch?v=kCPMph3cPA8}}
\\
On n'a plus de point de rebroussement contrairement \`{a} l'interpolation lin\'{e}aire par spline. 
\\\\
Interpolation quadratique par spline exemples Partie 1
\\
{\color{blue}
\url{https://www.youtube.com/watch?v=ES5zkgyMeAM}}
\\
Cette vid\'{e}o permet de prendre un exemple concret pour comprendre les deux vid\'{e}os pr\'{e}c\'{e}dentes.
\\\\
Interpolation quadratique par spline exemples Partie 2
\\
{\color{blue}
\url{https://www.youtube.com/watch?v=0NrMKwf4j0g}} 
\\
Chaque spline a un domaine de d\'{e}finition qui lui est propre.
\\
Cette vid\'{e}o se concentre essentiellement sur les domaines de d\'{e}finition des splines.
\\\\
Ces vid\'{e}os m'ont permis de comprendre l'interpolation par spline en g\'{e}n\'{e}ral mais je ne m'en sert pas directement dans le rapport.
\\\\
{\color{blue}
\url{https://www.math.u-bordeaux.fr/~pfischer/Teaching_files/cours.pdf}} (page 8 et 9)
\\
Ce document m'a permis de comprendre et de d\'{e}tailler la partie math\'{e}matiques du rapport concernant les constantes d'int\'{e}gration $a_i$ et $b_i$.
\\\\
{\color{blue}
\url{https://www.youtube.com/watch?v=kF-75RRvCbs}} 
\\
Cette vid\'{e}o m'a permis de comprendre et de d\'{e}tailler la partie math\'{e}matiques du rapport concernant l'algorithme Thomas.
\\\\
Pour le reste de la partie math\'{e}matiques, j'ai moi m\^{e}me d\'{e}taill\'{e} les calculs.
\\\\
{\color{blue}
\url{http://tex.stackexchange.com/}}
\\
Ce site m'a permis d'\'{e}crire mes parties en latex. (les packages \`{a} utiliser, la mise en page, cr\'{e}er des matrices...)

\end{document}