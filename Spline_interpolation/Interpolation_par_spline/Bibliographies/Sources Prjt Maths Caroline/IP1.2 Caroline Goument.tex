\documentclass{article}
\usepackage[T1]{fontenc}
\usepackage{graphicx}
\usepackage {subfigure}
\usepackage{color}
\usepackage{hyperref}
\usepackage{amsmath,amsfonts,amssymb}
\setlength{\parskip}{1em}
\usepackage{scrextend}
\usepackage{url}
\usepackage[margin=1in,footskip=0.25in]{geometry}

\title{Sources Bibliographiques}
\author{Caroline GOUMENT}
\date{D\'ecembre 2016}

\begin{document}

\maketitle

\vspace{10\baselineskip}
\begin{center}
\makeatother
\title{IP1.2 - L'INTERPOLATION PAR SPLINE}\\
\end{center}

\vspace{12\baselineskip}
\begin{center}
A l'attention de M.Gleyse\\
\date{2016-2017}
\end{center}

\newpage
\begingroup\raggedleft
Le document suivant est un cours pour des \'etudiants d'ASI3
\endgroup
\\\\
Les diff\'erents types d'interpolation
\\
{\color{blue}
\url{https://fr.scribd.com/doc/9233629/Methodes-numeriques-pour-l-ingenieur-interpolation-f-x}}
\\
Ce document m'a permis de mieux comprendre les interpolations polynomiales par spline cubique. On y trouve le principe de ce type d'interpolation, la d\'efinition, des illustrations, une explication math\'ematique et un court programme en pseudo-code sur les splines cubiques.
\\\\
Qu'est-ce qu'une spline ?
\\
{\color{blue}
\url{https://fr.wikipedia.org/wiki/Spline}}
\\ 
Qu'est-ce qu'une interpolation ?
\\
{\color{blue}
\url{https://fr.wikipedia.org/wiki/Interpolation_num%C3%A9rique}}
\\ 
Ces deux sources donnent des d\'efinitions des termes interpolation et spline et \'egalement des exemples. Elles peuvent servir pour d\'efinir les termes du sujet dans le rapport.
\\\\
Origine des splines
\\
{\color{blue}
\url{http://www.antigrain.com/research/bezier_interpolation/}}
\\
L'origine de l'interpolation par splines provient des courbes de B\'ezier. Le document permet donc de faire le lien entre l'interpolation par splines et les courbes de B\'ezier.
\\\\
Vid\'eo en anglais pour expliquer les splines cubiques
\\
{\color{blue}
\url{https://www.youtube.com/watch?v=f4iNbNRKZKU}}
\\
Cette vid\'{e}o explique simplement ce que sont les splines cubiques et comment les calculer.
\\\\
M\'ethode directe d'interpolation - Partie 1
\\
{\color{blue}
\url{https://www.youtube.com/watch?v=EtzlEA9MIwI}} 
\\
M\'ethode directe d'interpolation - Partie 2
\\
{\color{blue}
\url{https://www.youtube.com/watch?v=l-p7luIy1j4}} 
\\
Ces vid\'eos montrent avec un exemple concret comment fonctionne les splines cubiques, comment ils se calculent et \`a quoi ils peuvent servir (ex : calculer la distance parcourue par une fus\'ee et son acc\'el\'eration).
\\\\
Toutes ces documents (textes et vid\'eos) m'ont permis de comprendre les diff\'erentes m\'ethodes d'interpolation dans le cas g\'en\'eral mais je ne m'en suis pas servi dans le rapport.
Je me suis concentr\'ee sur les exemples et applications que l'on peut en faire.
\\\\
Le ph\'enom\`ene de Runge
\\
{\color{blue}
\url{http://johan.mathieu.free.fr/maths/doc_maths/oral_1_capes/phenomene_Runge_projet_licence.pdf}} (\`a partir de la page 19)
\\
Document pr\'esentant un exemple de fonction de Runge. Ceci m'a permis de d\'etailler le ph\'enom\`ene de Runge \`a partir d'un exemple concret.
\\\\
{\color{blue}
\url{http://www.sens-neuchatel.ch/bulletin/no34/art3-34.pdf}}
\\
J'ai pu trouver diverses id\'ees d'applications pour les m\'ethodes d'interpolation que nous \'etudions dans notre rapport.
\\\\
{\color{blue}
\url{https://www.math.u-psud.fr/~perrin/CAPES/geometrie/BezierDP.pdf}} 
\\
Ce document m'a permis de d\'etailler les applications pour lesquelles les interpolations sont utilis\'ees, en particulier pour la typographie.
\\\\
{\color{blue}
\url{http://w3.bretagne.ens-cachan.fr/math/people/gregory.vial/files/cs/interp.pdf}}
\\
Ce document contient des informations sur l'interpolation de la fonction exponentielle ainsi que sur le ph\'enom\`ene de Runge, informations qui m'ont \'et\'e utile pour d\'evelopper la partie sur les exemples.
\\\\
\newpage
{\color{blue}
\url{http://tex.stackexchange.com/}}
\\
Ce site m'a permis d'\'ecrire ma partie du rapport en latex. (les packages \`a utiliser, la mise en page, l'insertion de tableaux et d'images...)

\end{document}