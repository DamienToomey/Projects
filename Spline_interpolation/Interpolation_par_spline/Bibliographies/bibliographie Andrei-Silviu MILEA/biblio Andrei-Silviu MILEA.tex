\documentclass{article}
\usepackage[T1]{fontenc}
\usepackage{graphicx}
\usepackage {subfigure}
\usepackage{color}
\usepackage{hyperref}
\usepackage{amsmath,amsfonts,amssymb}
\setlength{\parskip}{1em}
\usepackage{scrextend}
\usepackage{url}
\usepackage[margin=1in,footskip=0.25in]{geometry}

\title{Bibliographie Personelle}
\author{Andrei-Silviu MILEA}
\date{Decembre 2016}

\begin{document}

\maketitle

\vspace{10\baselineskip}
\begin{center}
\makeatother
\title{Projet: L'INTERPOLATION PAR SPLINE}\\
\end{center}

\vspace{12\baselineskip}
\begin{center}
A l'attention de M.Gleyse\\
\date{2016-2017}
\end{center}

\newpage
\begingroup\raggedleft
Les sites et les vid\'{e}os consult\'{e}s sont en anglais, fran\c{c}ais et roumain:
\endgroup
\\\\
{\color{blue}
\url{http://asi.insa-rouen.fr/enseignement/siteUV/ananum/11interpol.pdf}\\
\url{https://fr.wikipedia.org/wiki/Spline}\\
\url{http://www.math.univ-metz.fr/~croisil/M1-0809/2}\\
\url{http://ccmuzzo.free.fr/perso/projets/Splines.pdf}\\
\url{http://lumimath.univ-mrs.fr/~jlm/cours/analnum/node30.html}}
\\\\
Les informations trouv\'{e}s sur ce site sont tr\`{e}s utiles et donne une id\'{e}e g\'{e}n\'{e}rales sur le calcul d'une
interpolation par spline cubique. Par contre, je trouve que ce site ne n'est pas assez d\'{e}taill\'{e} et que si
on le consid\`{e}re comme la seule source de documentation, il est vraiment difficile de se construire une
id\'{e}e \'{e}loquente sur l'interpolation cubique. De ce fait, il est plus dur de se baser sur ce site pour
comprendre l'id\'{e}e g\'{e}n\'{e}rale d'une interpolation par spline. En outre, il me semble que le site est un
peu ambigu et que le deuxi\`{e}me site que j'ai trouv\'{e} sur l'interpolation par spline cubique est plus
explicite, il a un structure plus professionnelle et je pense que c'est plus complet. Toutefois,
l'exemple pr\'{e}sent\'{e} m'ai aid\'{e} un peu \`{a} visualiser les grand lignes et de comparer la qualit\'{e}
d'information trouve dans des diff\'{e}rents sources.
\\\\
{\color{blue}
\url{http://www.giref.ulaval.ca/~afortin/mat17442/documents/splines.pdf}}
\\\\ 
La particularit\'{e} de ces deux sites qui a attir\'{e} mon attention est repr\'{e}sent\'{e} par la fa\c{c}on didactique dont
laquelle les informations sont repr\'{e}sent\'{e}s. Cela m'aide \`{a} bien comprendre l'id\'{e}e principale pr\'{e}sent\'{e}
dans les documents et de plus j'arrive \`{a} mieux visualiser l'int\'{e}r\^{e}t math\'{e}matique et la m\'{e}thodologie
de l'interpolation par spline. En outre, comme les deux sites pr\'{e}sentent l'interpolation par spline
d'ordre trois, je peux mettre les bases d'un cas particulier et de lui bien comprendre ce qui me
permettra de d\'{e}voiler plus facilement les m\'{e}thodes pour l'ordre n.
\\\\
{\color{blue}
\url{http://math.ubbcluj.ro/~tradu/sliderom/splineint.pdf}}
\\\\ 
Ce site est extr\^{e}mement important pour la bonne compr\'{e}hension du principe de l'interpolation par
spline. C'est un PDF qui contient un cours basique d'introduction \`{a} l'interpolation par spline qui est
utilis\'{e} par une grande et importante universit\'{e} roumaine. Le fait que le document soit r\'{e}dig\'{e} en
roumain m'aide \`{a} mieux comprendre tous les notions et la logique de base de cette m\'{e}thode. De plus,
la mani\`{e}re dont laquelle le document est structur\'{e} aide beaucoup la compr\'{e}hension et donne un
premier approche tr\`{e}s claire avec notre sujet.
\\\\
{\color{blue}
\url{http://www.sens-neuchatel.ch/bulletin/no34/art3-34.pdf}}
\\\\ 
J'ai t\'{e}l\'{e}charg\'{e} ce PDF pour mieux comprendre l'utilit\'{e} et le fonctionnement des courbes de splines.
C'est fortement utile pour la partie historique parce que les aspects les plus importants sont bien
pr\'{e}sent\'{e}s dans ce document. Puis, c'est tr\`{e}s int\'{e}ressant de voir la th\'{e}orie de splines quadratiques et
Lagrange comme information et pour enrichir la culture et les horizons de mes connaissances et du
projet aussi. Cependant, je me suis concentr\'{e} sur la th\'{e}orie de splines cubiques pour que je
comprenne les crit\`{e}res des choix des fonctions splines pour repr\'{e}senter diff\'{e}rents graphiques.
\\\\
{\color{blue}
\url{https://patrimoine.gadz.org/gadz/bezier.htm}}
\\\\
Le site a quelques informations tr\`{e}s utiles qui d\'{e}crivent bien la vie de Pierre B\'{e}zier. Premi\`{e}rement,
les \'{e}tudes suivies par B\'{e}zier pr\'{e}sent\'{e}s au d\'{e}but du site m'ont aid\'{e} \`{a} bien comprendre sa culture et ses sp\'{e}cialit\'{e}s, ce qui me donne des id\'{e}es plus claires sur le contexte dans lequel il a travaill\'{e}. Ensuite,
j'ai trouv\'{e} tr\`{e}s utile les lignes sur le programme UNISURF qui nous montre les bases de ce
programme qui maintenant est devenu fondamentale pour la conception des pi\`{e}ces des voitures.
\\\\
{\color{blue}
\url{http://www.math.u-psud.fr/~perrin/CAPES/geometrie/BezierDP.pdf}}
\\\\
Ce document montre comment on peut utiliser les courbes de Bezier, sp\'{e}cialement pour tracer des
paraboles et des cercles. Ce qui m'a int\'{e}ress\'{e} beaucoup ici est l'utilisation de l'interpolation par
fonctions splines pour la conception et la pr\'{e}sentation sous forme num\'{e}riques de cartes utilis\'{e}es soit
comme des simples images, soit pour les GPS. Cet exemple est tr\`{e}s utile pour la vie de chaque jour
est en d\'{e}couvrant que les fonctions splines peuvent avoir une utilit\'{e} pour le cr\'{e}er m'a beaucoup
impressionn\'{e}. De plus ce document est extr\^{e}mement utile pour l'impl\'{e}mentation des courbes de
B\'{e}zier en Geogebra, plus pr\'{e}cis\'{e}ment je l'ai utilis\'{e} pour v\'{e}rifier mes r\'{e}sultats et pour \^{e}tre sur si j'ai
des courbes coh\'{e}rents.
\\\\
{\color{blue}
\url{http://www.rasfoiesc.com/educatie/matematica/Curbe-si-suprafete-Bspline67.php}}
\\\\
R\'{e}dig\'{e} dans ma langue j'ai trouv\'{e} un site super int\'{e}ressant sur les B-splines qui aident \`{a} CAO,
conception assist\'{e} par ordinateur. Elle est utilis\'{e}e pour des surfaces en d\'{e}signant des fonctions
polynomiales par morceaux et des points fix\'{e}s appel\'{e}s points de control. On traite aussi les
principaux differences entres les courbes de Bezier et les B-splines, comme par exemple l'utilisation
seulement de la premi\`{e}re d\'{e}riv\'{e} pour Bezier tandis qu'on utilise des d\'{e}riv\'{e}es de plus grand ordre
pour les B-splines pour trouver des courbes plus lisses avec une plus haute continuit\'{e}. Cette partie
m'a d\'{e}voil\'{e} la vraie utilisation des courbes pour la conception des pi\`{e}ces des voitures et pas
seulement.
\\\\
{\color{blue}
\url{https://www.youtube.com/watch?v=7BuWXYPH9aU}} 
\\\\
J'ai utilis\'{e} ce vid\'{e}o pour bien me rappeler comment transformer des num\'{e}ros d\'{e}cimales en binaire et
je l'ai trouv\'{e} assez utile. Par contre, il fait seulement la d\'{e}composition et il ne va pas jusqu'au
moment quand on utilise aussi les bits de la machine pour bien repr\'{e}senter le num\'{e}ro. La technique
utilis\'{e} dans le vid\'{e}o m'ai fait me rappeler mais ce n'est pas tr\`{e}s bien fait pratiquement, donc j'ai
utilis\'{e} les exemples donn\'{e}s en classe. De plus sur YouTube j'ai trouv\'{e} un vid\'{e}o tr\`{e}s bonne qui
explique encore une fois tous les \'{e}tapes mais malheureuses j'ai perdu le lien et je devais me limiter \`{a}
ce vid\'{e}o qui est le seul qui traite les cas des d\'{e}cimales aussi.
\\\\
{\color{blue}
\url{https://fr.wikipedia.org/wiki/Courbe_de_B%C3%A9zier}} 
\\\\
J'ai utilis\'{e} ce site pour comprendre l'id\'{e}e g\'{e}n\'{e}rale de la construction math\'{e}matique des courbes de
B\'{e}zier. En fait, \`{a} l'aide des quelques informations trouv\'{e}es dans le cadre de ce site j'ai r\'{e}ussi de bien
d\'{e}velopper la partie math\'{e}matique de rapport pour ces courbes et en outre, j'ai r\'{e}ussi \`{a} trouver une
image tr\`{e}s suggestive que je vais m'en servir pour la pr\'{e}sentation pour expliquer l'id\'{e}e g\'{e}niale de
B\'{e}zier et quelle est la logique derri\`{e}re tous. J'ai utilis\'{e} plut\^{o}t les explications math\'{e}matiques,
cependant les autres informations sur ce site peuvent \^{e}tre tr\`{e}s utiles.
\\\\
{\color{blue}
\url{https://www.irif.fr/~carton/Enseignement/InterfacesGraphiques/MasterInfo/Cours/Swing/splines.html}} 
\\\\
Un site assez int\'{e}ressant qui m'a servi seulement pour la lecture et pour me remettre bien
l'information dans la m\'{e}moire. Par contre, je l'ai trouv\'{e} un peu incomplet et parfois ambigu mais
finalement je l'ai utilis\'{e} seulement pour la lecture.
\\\\
{\color{blue}
\url{http://www.math.univ-toulouse.fr/~cnegules/Article/beziers.pdf}}
\\\\
Le document ci-dessus est tr\`{e}s bon pour d\'{e}velopper le sujet des courbes de B\'{e}zier, je le trouve assez
bien organis\'{e} et l'information est assez claire et extr\^{e}mement int\'{e}ressante. En revanche, je ne l'ai pas
utilis\'{e} trop car j'ai r\'{e}ussi \`{a} rassembler plusieurs informations d'un autre PDF et des autres sites aussi.
Toutefois la lecture a \'{e}t\'{e} enrichissante.
\\\\
{\color{blue}
\url{http://www.duckworksmagazine.com/03/r/articles/splineducks/splineDucks.htm}} 
\\\\
Un site tr\`{e}s important car gr\^{a}ce \`{a} celui-l\`{a} j'ai r\'{e}ussi \`{a} me renseigner de la m\'{e}thode ancienne de
l'interpolation utilis\'{e} pour la construction des avions, bateaux et de quelques voitures. En fait, je
trouve que l'origine de l'utilisation des splines pour la CAO est repr\'{e}sent\'{e}e par cette m\'{e}thode
ancienne bas\'{e} sur des essais pratiques avec des morceaux de mat\'{e}riel plac\'{e}es entre les Ducks.
\\\\
{\color{blue}
\url{http://rocbo.lautre.net/bezier/pb-indus.htm}} 
\\
{\color{blue}
\url{https://imagescience.org/meijering/research/chronology/}} 
\\
{\color{blue}
\url{http://bigwww.epfl.ch/publications/meijering0201.pdf}} 
\\\\
Les deux sites au-dessus sont extr\^{e}mement importants pour l'histoire et les origines de
l'interpolation. Grace aux informations trouv\'{e}es dans le cadre de ces deux documents en langue
anglais et en utilisant des petits compl\'{e}ments des autres sites j'ai construit de z\'{e}ro toute la partie
historique de notre rapport. De plus, j'ai \'{e}largi mes horizons culturelles en me rendant compte de l'\'{e}volution de ce sujet indispensable pour une grand majorit\'{e} des op\'{e}rations du domaine
math\'{e}matique et non seulement.
\\\\
{\color{blue}
\url{https://en.wikipedia.org/wiki/B-spline}} 
\\
{\color{blue}
\url{https://fr.wikipedia.org/wiki/Spline#Spline_serr.C3.A9e}} 
\\
{\color{blue}
\url{https://en.wikipedia.org/wiki/Spline_(mathematics)#History}} 
\\
{\color{blue}
\url{https://fr.wikipedia.org/wiki/Interpolation_num%C3%A9rique}} 
\\\\
Ces quatre liens vers Wikip\'{e}dia j'ai utilis\'{e} notamment pour des informations compl\'{e}mentaires et
\'{e}galement pour v\'{e}rifier les donn\'{e}es trouv\'{e}es \`{a} l'aide des autres sources. La partie concernant
l'historique des splines est tr\`{e}s utile pour mettre les bases de l'apparition des splines et leur utilit\'{e}
dans le domaine math\'{e}matique en particulier. On sait que ce source n'a pas toujours une bonne
cr\'{e}dibilit\'{e} car elle est une encyclop\'{e}die ouverte aux grand publique qui peut \^{e}tre facilement modifi\'{e},
ce qui peut entrainer des fautes ou des informations incompl\`{e}tes. Toutefois, j'ai v\'{e}rifi\'{e} tous les
donnes collectes avec plusieurs sources et je me suis rendu compte que les articles sont tr\`{e}s bien
r\'{e}dig\'{e}es et donnes des informations coh\'{e}rents. De plus, j'ai re\c{c}u un renseignement comme que les
articles de Wikip\'{e}dia sont bien v\'{e}rifi\'{e}s et valid\'{e} par une certaine commission. Donc, en gros il me
semble que les articles sont utiles et ils m'ont aid\'{e} \`{a} enrichir le projet et ma culture g\'{e}n\'{e}rale aussi.
\\\\
{\color{blue}
\url{http://tex.stackexchange.com/}} 
\\\\
Notamment utilis\'{e} pour \'{e}crire en TexMaker et pour avoir des reinsegnements compl\'{e}mentaires.
\\\\

\end{document}