\documentclass[a4paper,12pt]{article}
\usepackage[utf8]{inputenc}
\usepackage[T1]{fontenc}
\usepackage[francais]{babel}
\usepackage{amsmath}
\usepackage{color}
\usepackage{hyperref}

\title{Décomposition de la variance}
\date{}

\begin{document}
\maketitle

\vspace{10\baselineskip}
\begin{center}
	\author{ABOUZAID Mehdi \\ TOOMEY Damien}
\end{center}

\newpage
\tableofcontents

\newpage
\section{Notations}
\color{white} ligne inutile\\
\color{black} $ c $: nombre de groupes \\
$ r_i $ : effectif d'un groupe, $ i= 1...c$ \\
$ n $ : effectif total \\
\\ 
Moyenne globale (tous groupes confondus) :
\[ \bar{y}=\frac{1}{c} \cdot \sum_{i=1}^{c} \bar{y}_i\]
\\
Moyenne d'un groupe :
\[ \bar{y}_i=\frac{1}{r_i} \cdot \sum_{j=1}^{r_i} y_{ij} \]

\section{Modèle}
\[ y_{ij}=\mu_i + \epsilon_i \]
\\
Dans cette démonstration, pour rester général, on considère que les groupes n'ont pas forcément les mêmes effectifs.

\section{Décomposition de la somme des carrés des écarts}

\begin{align*}
& \text{SCE} = \text{somme des carrés des écarts} \\
& \;\;\;\;\;\;\; = \sum_{i=1}^{c} \sum_{j=1}^{r_i}  (y_{ij}-\bar{y})^2 \\
& \;\;\;\;\;\;\; = \sum_{i=1}^{c} \sum_{j=1}^{r_i}  (y_{ij}-\bar{y}_i+\bar{y}_i-\bar{y})^2 \\
& \;\;\;\;\;\;\; = \sum_{i=1}^{c} \sum_{j=1}^{r_i}  (y_{ij}-\bar{y}_i)^2 + 2 \cdot \sum_{i=1}^{c} \sum_{j=1}^{r_i} (y_{ij}-\bar{y}_i) \cdot (\bar{y}_i-\bar{y}) + \sum_{i=1}^{c} \sum_{j=1}^{r_i} (\bar{y}_i-\bar{y})^2 \quad \textbf{(1)}\\
& \;\;\;\;\;\;\; = \sum_{i=1}^{c} \sum_{j=1}^{r_i}  (y_{ij}-\bar{y}_i)^2 + \sum_{i=1}^{c} \sum_{j=1}^{r_i} (\bar{y}_i-\bar{y})^2 \quad \textbf{(2)} \\
& \;\;\;\;\;\;\; = \sum_{i=1}^{c} \sum_{j=1}^{r_i}  (y_{ij}-\bar{y}_i)^2 + \sum_{i=1}^{c} r_i \cdot (\bar{y}_i-\bar{y})^2 \\
& \qquad \qquad \; \text{Intra-Classe} \qquad \quad \text{Inter-Classes} \\
\end{align*}

Passage des lignes \textbf{(1)} à \textbf{(2)} :

\begin{align*}
& \quad \sum_{i=1}^{c} \sum_{j=1}^{r_i} (y_{ij}-\bar{y}_i) \cdot (\bar{y}_i-\bar{y}) \\
& = \sum_{i=1}^{c} \sum_{j=1}^{r_i} (y_{ij} \cdot \bar{y}_i - y_{ij} \cdot \bar{y} - \bar{y}_i^2 + \bar{y}_i \cdot \bar{y}) \\
& = \sum_{i=1}^{c} \sum_{j=1}^{r_i} (\bar{y}_i \cdot (y_{ij}-\bar{y}_i)-\bar{y} \cdot (y_{ij}-\bar{y}_i)) \\
& = \sum_{i=1}^{c} (\bar{y}_i \cdot \sum_{j=1}^{r_i} (y_{ij}-\bar{y}_i)-\bar{y} \cdot \sum_{j=1}^{r_i} (y_{ij}-\bar{y}_i))
\end{align*}

or

\begin{align*}
& \quad \sum_{j=1}^{r_i} (y_{ij}-\bar{y}_i) \\
& = (\sum_{j=1}^{r_i} y_{ij}) - \bar{y}_i \cdot (\sum_{j=1}^{r_i} 1) \\
& = (\sum_{j=1}^{r_i} y_{ij}) - \bar{y}_i \cdot r_i \\
& = (\sum_{j=1}^{r_i} y_{ij}) - \frac{1}{r_i} \cdot \sum_{f=1}^{r_i} y_{if} \cdot r_i \\
& = (\sum_{j=1}^{r_i} y_{ij}) - \sum_{f=1}^{r_i} y_{if} \\
& = 0
\end{align*}

En effet, la moyenne des écarts à la moyenne est toujours nulle. \\\\
Ainsi,

\[ 2 \cdot \sum_{i=1}^{c} \sum_{j=1}^{r_i} (y_{ij}-\bar{y}_i) \cdot (\bar{y}_i-\bar{y}) = 0 \]
\\

\section{Décomposition de la variance}

\begin{align*}
& \Rightarrow \text{Variance} = \frac{\text{SCE}}{n} = \frac{1}{n} \cdot \sum_{i=1}^{c} \sum_{j=1}^{r_i}  (y_{ij}-\bar{y}_i)^2 + \frac{1}{n} \cdot \sum_{i=1}^{c} r_i \cdot (\bar{y}_i-\bar{y})^2 \\
& \qquad \qquad \qquad \qquad \qquad \; \text{Variance Intra-Classe} \;\;\;\; \text{Variance Inter-Classes} \\
& \qquad \qquad \qquad \qquad \qquad \quad \text{Variance résiduelle} \qquad \; \text{Variance expliquée}
\end{align*}
\\
Par la suite, on considérera que l'effectif de chaque groupe est identique ($ r_i = r, \;\; \forall i=1...c $).

\section{Bibliographie}

\begin{itemize}
\item[•] Wikipedia
\item[] \url{https://fr.wikipedia.org/wiki/Analyse\_de\_la\_variance}
\end{itemize}

\end{document}
