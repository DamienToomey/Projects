\subsection{Types Abstraits de Donn�es (TAD)}
\subsubsection{TAD Mot}

\begin{tad}
  \tadNom{Mot}
  \tadDependances{\caractere, \chaine, \naturelNonNul, \booleen, Ensemble}
  \begin{tadOperations}{}
    \tadOperation{longueur}{\tadUnParam{Mot}}{\tadUnParam{\naturel}}
    \tadOperationAvecPreconditions{obtenirIemeCaractere}{\tadDeuxParams{Mot}
      {\naturelNonNul}}{\tadUnParam{\caractere}}
    \tadOperationAvecPreconditions{fixerIemeCaractere}{\tadTroisParams{Mot}
      {\naturelNonNul}{\caractere}}{\tadUnParam{Mot}}
    \tadOperationAvecPreconditions{remplacerLettre}{\tadTroisParams{Mot}
      {\naturelNonNul}{\caractere}}{\tadUnParam{Mot}}
    \tadOperationAvecPreconditions{inverserDeuxLettresConsecutives}
    {\tadDeuxParams{Mot}{\naturelNonNul}}{\tadUnParam{Mot}}
    \tadOperationAvecPreconditions{supprimerLettre}{\tadDeuxParams{Mot}
      {\naturelNonNul}}{\tadUnParam{Mot}}
    \tadOperationAvecPreconditions{insererLettre}{\tadTroisParams{Mot}
      {\caractere}{\naturelNonNul}}{\tadUnParam{Mot}}
    \tadOperation{obtenirMotEntreeStandard}{\tadDeuxParams{Mot}{\entier \repnote}}
    {\tadTroisParams{\entier \repnote}{Mot}{\booleen}}
    \tadOperation{fixerLaChaine}{\tadDeuxParams{Mot}{\chaine}}{\tadUnParam{Mot}}
    \tadOperationAvecPreconditions{obtenirLaChaine}{\tadUnParam{Mot}}
    {\tadUnParam{\chaine}}
    \tadOperation{fixerLongueur}{\tadDeuxParams{Mot}{\naturel}}{\tadUnParam{Mot}}
    \tadOperation{obtenirLongueur}{\tadUnParam{Mot}}{\tadUnParam{\naturel}}
    \tadOperation{sontEgaux}{\tadDeuxParams{Mot}{Mot}}{\tadUnParam{\booleen}}
    \tadOperation{estUneLettre}{\tadUnParam{\caractere}}{\tadUnParam{\booleen}}
  \end{tadOperations}

  \begin{tadAxiomes}
    \tadAxiome{supprimer(insererLettre(mot, c, i), i)=mot}
    \tadAxiome{inverserDeuxLettresConsecutives(inverserDeuxLettresConsecutives(
      mot, pos), pos)=mot}
  \end{tadAxiomes}

  \begin{tadPreconditions}{}
    \tadPrecondition{obtenirIemeCaractere(mot, pos)}{1 $\leq$ pos
      $\leq$ obtenirLongueur(mot)}
    \tadPrecondition{fixerIemeCaractere(mot, pos, c)}{1 $\leq$ pos
      $\leq$ obtenirLongueur(mot)+1}
    \tadPrecondition{remplacerLettre(mot, pos, lettre)}{1 $\leq$ pos
      $\leq$ obtenirLongueur(mot)}
    \tadPrecondition{inverserDeuxLettresConsecutives(mot, pos)}{1
      $\leq$ pos $\leq$ obtenirLongueur(mot)-1}
    \tadPrecondition{supprimerLettre(mot, pos)}{(1 $\leq$ pos $\leq$
      obtenirLongueur(mot)) et (obtenirLongueur(mot) $\geq$ 2)}
    \tadPrecondition{insererLettre(mot, c, pos)}{1 $\leq$ pos $\leq$
      obtenirLongueur(mot)+1}
    \tadPrecondition{obtenirLaChaine(mot)}{obtenirLongueur(mot)>=0}
  \end{tadPreconditions}
\end{tad}

\subsubsection{TAD Dictionnaire}

\begin{tad}
  \tadNom{Dictionnaire}
  \tadParametres{Mot}
  \tadDependances{\booleen, \naturel}
  \begin{tadOperations}{}
    \tadOperation{creerDico}{}{\tadUnParam{Dictionnaire}}

    \tadOperation{estVide}{\tadUnParam{Dictionnaire}}{\tadUnParam{\booleen}}

    \tadOperation{insererMot}{\tadDeuxParams{Dictionnaire}{Mot}}
    {\tadUnParam{Dictionnaire}}

    \tadOperation{estPresent}{\tadDeuxParams{Dictionnaire}{Mot}}
    {\tadUnParam{\booleen}}

    \tadOperation{sauvegarderDico}{\tadDeuxParams{\chaine}{Dictionnaire}}
    {\tadUnParam{Fichier Texte}}

    \tadOperation{chargerDico}{\tadDeuxParams{\chaine}{Fichier Texte}}
    {\tadUnParam{Dictionnaire}}
  \end{tadOperations}

  \begin{tadAxiomes}
    \tadAxiome{insererMot(mot, insererMot(mot, dico))=insererMot(mot,dico)}

    \tadAxiome{estPresent(mot, insererMot(mot, dico)}
  \end{tadAxiomes}
\end{tad}

\subsubsection{TAD CorrecteurOrthographique}

\begin{tad}
  \tadNom{CorrecteurOrthographique}
  \tadParametres{Mot, Dictionnaire}
  \tadDependances{\booleen, Ensemble}
  \begin{tadOperations}{}
    \tadOperation{correcteurOrthographique}{}
    {\tadUnParam{CorrecteurOrthographique}}

    \tadOperationAvecPreconditions{proposerCorrection}
    {\tadDeuxParams{CorrecteurOrthographique}}{\tadUnParam{Ensemble<Mot>}}

    \tadOperationAvecPreconditions{estBienOrthographie}
    {\tadDeuxParams{CorrecteurOrthographique}}{\tadUnParam{\booleen}}

    \tadOperation{obtenirMot}{\tadUnParam{CorrecteurOrthographique}}
    {\tadUnParam{Mot}}

    \tadOperation{obtenirDictionnaire}{\tadUnParam{CorrecteurOrthographique}}
    {\tadUnParam{Dictionnaire}}

    \tadOperation{fixerMot}{\tadDeuxParams{CorrecteurOrthographique}{Mot}}
    {\tadUnParam{CorrecteurOrthographique}}

    \tadOperation{fixerDictionnaire}{\tadDeuxParams{CorrecteurOrthographique}
      {Dictionnaire}}{\tadUnParam{CorrecteurOrthographique}}
  \end{tadOperations}

  \begin{tadPreconditions}{}
    \tadPrecondition{proposerCorrection(co)}{non estBienOrthographie(co)}
  \end{tadPreconditions}
\end{tad}

\subsubsection{TAD Arbre n-aire}

Pour repr�senter notre TAD Dictionnaire nous avons choisis d'utiliser
un arbre n-aire. Nous allons maintenant d�tailler ce TAD.\\

\begin{tad}
  \tadNom{AbreN}
  \tadDependances{\booleen, \chaine, \caractere, Ensemble}
  \begin{tadOperations}{}

    \tadOperation{creerArbreNonInit}{}{\tadUnParam{ArbreN}}
    \tadOperation{estVide}{\tadUnParam{ArbreN}}{\tadUnParam{\booleen}}
    \tadOperationAvecPreconditions{obtenirBool}{\tadUnParam{ArbreN}}
    {\tadUnParam{\booleen}}
    \tadOperationAvecPreconditions{obtenirChar}{\tadUnParam{ArbreN}}
    {\tadUnParam{\caractere}}
    \tadOperationAvecPreconditions{fixerBool}{\tadDeuxParams{ArbreN}
      {\booleen}}{\tadUnParam{ArbreN}}
    \tadOperationAvecPreconditions{fixerChar}{\tadDeuxParams{ArbreN}
      {\caractere}}{\tadUnParam{ArbreN}}
    \tadOperationAvecPreconditions{fixerFils}{\tadDeuxParams{ArbreN}{ArbreN}}
    {\tadUnParam{ArbreN}}
    \tadOperationAvecPreconditions{fixerFrere}{\tadDeuxParams{ArbreN}{ArbreN}}
    {\tadUnParam{ArbreN}}
    \tadOperationAvecPreconditions{obtenirFils}{\tadUnParam{ArbreN}}
    {\tadUnParam{ArbreN}}
    \tadOperationAvecPreconditions{obtenirFrere}{\tadUnParam{ArbreN}}
    {\tadUnParam{ArbreN}}
  \end{tadOperations}

  \begin{tadPreconditions}{}

    \tadPrecondition{obtenirBool(arbre)}{non estVide(arbre)}
    \tadPrecondition{obtenirChar(arbre)}{non estVide(arbre)}
    \tadPrecondition{fixerBool(arbre, bool)}{non estVide(arbre)}
    \tadPrecondition{fixerChar(arbre, car)}{non estVide(arbre)}
    \tadPrecondition{fixerFils(arbre, fils)}{non estVide(arbre)}
    \tadPrecondition{fixerFrere(arbre, frere)}{non estVide(arbre)}
    \tadPrecondition{obtenirFils(arbre)}{non estVide(arbre)}
    \tadPrecondition{obtenirFrere(arbre)}{non estVide(arbre)}
  \end{tadPreconditions}
\end{tad}

\newpage
\subsection{Analyse descendante}

Notre analyse descendante se trouve sur les quatre pages suivantes. Du
fait de sa taille, les figures ne sont pas tr�s nettes. Nous vous
invitons donc � regarder les images AnalyseDescendanteMAIN-90,
AnalyseDescendanteMettreMotDansSDDdico-90, AnalyseDescendanteCorrigerMot-90,\\
AnalyseDescendanteobtenirMotEntreeStandard-90 au format jpg dans le
r�pertoire /rapport/images. Vous pouvez aussi cliquer
\href{https://drive.google.com/file/d/1KbCsgzSyPy5rJcjGTQRZmwdG7l1mG2OS/view?usp=sharing}{\emph{ici}},
pour voir, en ligne, l'image de l'analyse descendante du programme principale,
\href{https://drive.google.com/file/d/1I7NCatB1OX_qTifBynEafyC2xIsoMoRn/view?usp=sharing}{\emph{ici}}
pour celle correspondant � 'mettreMotDansSDDdico',
\href{https://drive.google.com/file/d/14eHbYWqVv2qjTY-rC_iW8tHbd5dhlxcR/view?usp=sharing}{\emph{ici}}
pour celle correspondant � `corrigerMot', ou encore
\href{https://drive.google.com/file/d/15DQ746vTvA2d2KU2f6ZqBqg-CSaY8RHV/view?usp=sharing}{\emph{ici}}
pour celle correspondant � 'obtenirMotEntreeStandard'.


\newpage
 \begin{center}
   \pgfimage[height=\textheight,interpolate=true]
   {images/AnalyseDescendanteMAIN-90}
 \end{center}

 \newpage
 \begin{center}
   \pgfimage[height=\textheight,interpolate=true]
   {images/AnalyseDescendanteobtenirMotEntreeStandard-90.jpg}
 \end{center}

 \newpage
 \begin{center}
   \pgfimage[height=\textheight,interpolate=true]
   {images/AnalyseDescendanteCorrigerMot-90.jpg}
 \end{center}

 \newpage
 \begin{center}
   \pgfimage[height=\textheight,interpolate=true]
   {images/AnalyseDescendanteMettreMotDansSDDdico-90.jpg}
 \end{center}

 \clearpage