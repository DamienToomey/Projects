\documentclass{article}
\usepackage[T1]{fontenc}
\usepackage{graphicx}
\usepackage {subfigure}
\usepackage{color}
\usepackage{hyperref}
\usepackage{amsmath,amsfonts,amssymb}
\setlength{\parskip}{1em}
\usepackage{scrextend}
\usepackage{url}
\usepackage[margin=1in,footskip=0.25in]{geometry}

\title{Sources Bibliographiques}
\author{Olivia BRIODEAU}
\date{D\'{e}cembre 2016}

\begin{document}

\maketitle

\vspace{10\baselineskip}
\begin{center}
\makeatother
\title{L'INTERPOLATION PAR SPLINE}\\
\end{center}

\vspace{12\baselineskip}
\begin{center}
A l'attention de M.Gleyse\\
\date{2016-2017}
\end{center}

\newpage
\begingroup\raggedleft
{\color{blue}
\url{http://www.math.univ-metz.fr/~croisil/M1-0809/2}}
\endgroup
\\
Ce document donne une d\'{e}finition math\'{e}matique des splines ainsi que des m\'{e}thodes de d\'{e}termination des splines cubiques. Les conditions de bord (Hermite, "naturelles", p\'{e}riodiques) n\'{e}cessaires pour cette interpolation sont expliqu\'{e}es bri\`{e}vement. Il nous informe sur l'interpr\'{e}tation m\'{e}canique de la spline cubique et l' illustre par des graphiques qui mettent en \'{e}vidence la pr\'{e}cision de ce mod\`{e}le.
\\\\
{\color{blue}
\url{http://www.ulb.ac.be/di/map/gbonte/calcul/math31_6_1_dia.pdf}} 
\\
Ce document d\'{e}taille de fa\c{c}on tr\`{e}s compl\`{e}te, th\'{e}oriquement puis de fa\c{c}on pratique, le calcul math\'{e}matique d'une spline cubique. L'avantage des splines est aussi expliqu\'{e} : en augmentant n on augmente le nombre de morceaux et non le degr\'{e} des polyn\^{o}mes. Donc l'approximation reste correcte plus loin du n\oe{}ud contrairement \`{a} l'interpolation avec un polyn\^{o}me de degr\'{e} \'{e}lev\'{e}.
\\\\
{\color{blue}
\url{http://asi.insa-rouen.fr/enseignement/siteUV/ananum/11interpol.pdf}}
\\
Ce document montre diff\'{e}rentes fa\c{c}ons d'approximer des fonctions et leurs utilit\'{e}s. Les principes d'interpolation polynomiale et d'interpolation par spline y sont d\'{e}velopp\'{e}s . La m\'{e}thode d'interpolation par spline est d\'{e}finie et illustr\'{e}e pour comprendre la meilleure pr\'{e}cision qu'elle offre par rapport \`{a} l'interpolation polynomiale. Ce document contient aussi une m\'{e}thode pour d\'{e}terminer les splines et un exemple pour illustrer les avantages d'une telle interpolation.
\\\\
{\color{blue}
\url{http://www.labri.fr/perso/schlick/simg/cours10.pdf}} 
\\
Ce diaporama sur les courbes de B\'{e}zier et les courbes B-splines \'{e}voque bri\`{e}vement les avantages et inconv\'{e}nients des courbes de B\'{e}zier.
\\\\
{\color{blue}
\url{http://pulsar.webshaker.net/2012/08/29/les-courbes-de-bezier-1/ }} 
\\
Cette page internet d\'{e}taille les avantages des courbes de B\'{e}zier : la possibilit\'{e} de dessiner des formes complexes et la simplicit\'{e} des transformations.
\\\\
{\color{blue}
\url{http://perso.univ-lyon1.fr/jean-claude.iehl/Public/educ/ENS/chap6_Courbes.pdf }} 
\\
Ce document pr\'{e}sente des propri\'{e}t\'{e}s des courbes de B\'{e}zier.
\\\\
{\color{blue}
\url{http://ccmuzzo.free.fr/perso/projets/Splines.pdf }} 
\\
Ce rapport explique l'int\'{e}r\^{e}t de la m\'{e}thode d'interpolation par splines par rapport \`{a} d'autres m\'{e}thodes d'interpolation gr\^{a}ce \`{a} sa pr\'{e}cision et sa simplicit\'{e} d'utilisation.
\\\\
{\color{blue}
\url{http://www.giref.ulaval.ca/~afortin/mat17442/documents/splines.pdf}} 
\\
Ce document diff\'{e}rencie les m\'{e}thodes d'interpolation par spline en expliquant leurs sp\'{e}cificit\'{e}s et utilit\'{e}s.
\\\\
{\color{blue}
\url{http://homepages.ulb.ac.be/~majansen/teaching/INFO-F-205/diapositives05interpolation_4.pdf }} 
\\
Ce diaporama permet de comprendre l'utilit\'{e} des splines et leur utilisation dans les logiciels de dessin mais aussi les limitations des splines d'interpolation.
\\\\
{\color{blue}
\url{http://www.lifl.fr/~grisoni/IVI/Cours1Splines.pdf}} 
\\
Ce document est compos\'{e} d'un bref historique sur les courbes et surfaces splines qui pr\'{e}sente l'id\'{e}e \`{a} partir de laquelle a \'{e}t\'{e} d\'{e}velopp\'{e} les splines.
\\\\
{\color{blue}
\url{http://www-groups.dcs.st-and.ac.uk/history/Biographies/Schoenberg.html}} 
\\
Cette page internet en langue anglaise retrace la biographie de Isaac Jacob Schoenberg, scientifique qui a r\'{e}alis\'{e} de nombreux travaux  contribuant grandement au d\'{e}veloppement des splines.
\\\\
{\color{blue}
\url{http://tex.stackexchange.com/}} 
\\
Ce site m'a permis d'\'{e}crire des parties du rapport en latex.
\end{document}