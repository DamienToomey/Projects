\documentclass{article}
\usepackage[T1]{fontenc}
\usepackage{graphicx}
\usepackage {subfigure}
\usepackage{color}
\usepackage{hyperref}
\usepackage{amsmath,amsfonts,amssymb}
\setlength{\parskip}{1em}
\usepackage{scrextend}
\usepackage{url}
\usepackage[margin=1in,footskip=0.25in]{geometry}

\title{Sources Bibliographiques}
\author{Bozhidar PALASHEV}
\date{Decembre 2016}

\begin{document}

\maketitle

\vspace{10\baselineskip}
\begin{center}
\makeatother
\title{L'INTERPOLATION PAR SPLINE}\\
\end{center}

\vspace{12\baselineskip}
\begin{center}
A l'attention de M.Gleyse\\
\date{2016-2017}
\end{center}

\newpage
\begingroup\raggedleft
Les sites suivantes sont trouv\'es au cours de pr\'eparation du rapport sur l'interpolation par spline.
\endgroup
\\
\\
{\color{blue}
\url{https://fr.wikipedia.org/wiki/Spline}}
\\
{\color{blue}
\url{https://fr.wikipedia.org/wiki/Interpolation_numerique}}
\\ 
{\color{blue}
\url{https://fr.wikipedia.org/wiki/NURBS}}
\\ 
Selon moi, Wikipedia reste une site o\`u on peut trouver des information sur tout les domain possible.
J'avoue que c'est pas suffisant, mais c'est une bonne point de d\'epart. Aussi je trouve que l'information pour les NURBS est tr\`es bien d\'etaill\'ee. 
\\\\
{\color{blue}
\url{http://www.math.univ-metz.fr/~croisil/M1-0809/2}}\\
Je trouve que c'est bien expliquer la plus part des chose et les graphes aide beaucoup pour comprendre les erreurs par rapport aux points utilis\'es. Par contre il y a des moment o\`u j'ai du mal \`a comprendre.\\
\\
{\color{blue}
\url{http://www.geos.ed.ac.uk/~yliu23/docs/lect_spline.pdf}}\\
C'est en anglais, mais il est bien expliquer l'int\'er\^et et la fa\c con de faire l'interpolation par spline des fonctions. Ici aussi les graphiques sont bien faites.\\
\\ 
{\color{blue}
\url{http://www.giref.ulaval.ca/~afortin/mat17442/documents/splines.pdf}}\\
Le document est utile pour comprendre l'interpolation par spline, car il commence ses explications
de loin pour arriver au but et de cette fa\c con on a une meilleure conception.
\\\\
{\color{blue}
\url{http://www.ulb.ac.be/di/map/gbonte/calcul/math31_6_1_dia.pdf}}\\
Les graphiques pour chaque degr\'e des fonctions d'interpolotion par spline ont \'et\'e tr\`es utiles pour la comprehention de l'evolution des fonctions des splines.
\\\\
{\color{blue}
\url{http://www.sens-neuchatel.ch/bulletin/no34/art3-34.pdf}}\\
Ce document, \'etait le principale source d'information sur les bases de B-splines. Avec les exemples accompagn\'es avec des graphiques ont \'et\'e tr\`es precis et ont beaucoup aid\'e la construction de la partie sur les B-splines.
\\\\ 
{\color{blue}
\url{http://www.math.univ-metz.fr/~croisil/M1-0809/2}}\\
Gr\^ace \`a ce diaporama on peut se rendre compte pour les erreurs possibles en fonction du nombre des points de contr\^ole utilis\'e.
\\\\ 
{\color{blue}
\url{https://team.inria.fr/virtualplants/files/2014/09/cours_NURBS.pdf}}\\
Ce fichier permet d'avoir les connaissances initiales pour comprendre les courbes NURBS.
\\\\ 

\end{document}